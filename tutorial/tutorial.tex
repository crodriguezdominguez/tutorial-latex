\documentclass[12pt]{book} % "book",  "article",  "report" ... 
\usepackage[utf8]{inputenc} % para que se acepten acentos, ñ, etc.

\usepackage[spanish]{babel} % traduce los textos autogenerados al español
\usepackage{url} % permite poner urls en nuestro texto
\usepackage[dvipsnames]{xcolor} % uso de colores básicos y avanzados
\usepackage{amsmath} % ecuaciones y operadores matemáticos avanzados
\usepackage{amssymb} % símbolos matemáticos
\usepackage{graphicx} % posibilidad de trabajar con imágenes
\usepackage{array} % gestión avanzada de tablas
\usepackage{csvsimple} % permite crear tablas a partir de CSV
\usepackage{multirow} % combinación de filas y columnas en tablas
\usepackage{listings} % permite incluir listados de código (cap. 2)
\usepackage{subcaption} % permite crear "subfiguras"
\usepackage{wrapfig} % tablas o imágenes envueltas en texto
\graphicspath{ {imagenes/} } % carpeta donde meteremos las imágenes

\usepackage{hyperref} % crear hipervínculos automáticos en índices, enlaces, etc. Se suele importar como ÚLTIMO PAQUETE para que funcione apropiadamente.

% el comando \LaTeX incluye el logo de LaTeX en el documento.
\title{Tutorial de \LaTeX}
\author{Carlos Rodríguez Domínguez}
\date{Mayo 2021}

\begin{document}

\frontmatter % en el tipo de documento "book", esto indica que aquí comienza el título, índices, etc. Observar numeración de páginas.
\maketitle % muestra el título
\tableofcontents % crea un índice

% al incluir un * tras un chapter, section, etc., este no se numerará y no se incluirá por defecto en el índice.
% esto es común para agradecimientos, abstract, etc.
\chapter*{Notas iniciales}

El objetivo de este documento es demostrar las funcionalidades más básicas y comunes de \LaTeX.

Este documento debe leerse compaginando la lectura del código fuente con el PDF generado por \LaTeX. Es muy recomendable mantener en pantalla ambos contenidos en todo momento, y tratar de entender  cómo se generan los contenidos del documento PDF a partir de los comandos de \LaTeX.

Se asume que el lector de este documento ya posee una instalación de \LaTeX y un editor de dicho lenguaje.

\mainmatter % en el tipo de documento "book", esto indica que aquí comienza el contenido principal del documento

\chapter{Sintáxis básica}

\LaTeX permite escribir textos de manera cómoda y sencilla.  Nos va a ayudar a darle un estilo profesional a todo lo que escribamos, y a autogenerar índices,  capítulos,  secciones, etc.

Al principio \LaTeX es un poco abrumador, así que trataremos de aprenderlo paso a paso.

Para ello, leeremos este documento con el código a su lado. De esta manera observaremos cómo se podría escribir un documento similar a este.

\section{Comentarios} % Esto puede ser un comentario. Podemos ver que en el pdf generado no aparece

Dentro de nuestro código podemos dejar indicaciones para recordar ciertos detalles o dar instrucciones a nuestros compañeros. Los comentarios jamás se visualizarán en el documento final. Los comentarios debemos comenzarlos por el símbolo \%.

\section{Comandos}

Los comandos son instrucciones para \LaTeX que permiten formatear nuestro documento como deseemos. Siempre comienzan por la letra \textbackslash.

Los comandos tienen parámetros y opciones. Los parámetros son aquellos contenidos que toma el comando para poder producir los resultados que esperamos. Se ponen entre \{ \}.

Las opciones son parámetros no obligatorios que habitualmente alteran la manera de actuar del comando principal. Se ponen entre [ ]. Por ejemplo, permiten cambiar el tamaño por defecto de una figura (se verá más adelante).

\subsection{Comandos personalizados}

% mirar el resultado de toda esta sección en el PDF
Se pueden crear nuevos comandos en el preámbulo mediante el comando \textbf{\textbackslash newcommand\{\textbackslash \emph{nombre}\}[\emph{núm. parámetros}]\{\emph{definición}\}}.

De esta forma, si pusiésemos \textbackslash newcommand\{\textbackslash hello\}[1]\{Hola \{\#1\}\} podríamos usar el comando \textbackslash hello\{Mundo\}, cuyo resultado sería ``Hola Mundo''.

Asimismo, podemos cambiar los comandos de \LaTeX o de cualquier paquete usando el comando \textbf{\textbackslash renewcommand}, con la misma sintáxis que \textbackslash newcommand.

\section{Símbolos reservados y caracteres especiales}

% en esta sección mirar detenidamente el código y el resultado en el PDF
Para usar dentro de nuestros textos los símbolos que utiliza \LaTeX para dar estilo a los textos o especificar el preámbulo, que son: \# \$ \% \& \{ \} \_ \~{} \^{} \textbackslash, debemos anteponer en nuestro código a dichos símbolos el símbolo \textbackslash. El propio símbolo \textbackslash es un tanto especial, y hay que escribirlo mediante el comando \textbackslash textbackslash.

Por otro lado, si deseamos insertar caracteres especiales lo tendremos que hacer mediante comandos. Un amplio listado de dichos comandos está disponible en: \url{https://en.wikibooks.org/wiki/LaTeX/Special_Characters}. 

\section{Preámbulo}

Los documentos \LaTeX incluyen códigos para que se puedan generar los documentos finales adecuadamente.  Todo lo que se escribe en el código antes de \textbf{\textbackslash begin\{document\}} se le conoce como el preámbulo.

El preámbulo incluye información sobre los paquetes que necesitamos usar para generar el documento final, el tipo de documento que estamos escribiendo, tamaño de letra, estilo básico, etc.

Además, en el preámbulo se especifica el título, autor y fecha del documento.

\section{Tipos de documentos}

\LaTeX es capaz de aplicar estilos predefinidos a diversos tipos de documentos: libros, artículos, etc. Dichos estilos se aplican en el preámbulo, dentro del comando \textbf{\textbackslash documentclass}.

Entre los estilos de los que disponemos por defecto (se recomienda probarlos) están: \emph{article}, \emph{report}, \emph{book}, etc. \LaTeX permite definir nuestros propios estilos (muy avanzado), o bien incluir estilos de terceros (muy habitual en congresos/revistas).

Asimismo, el comando \textbf{\textbackslash documentclass} permite especificar el tamaño del papel (\emph{a4paper}, \emph{a5paper}, \emph{executivepaper}, etc.), de la tipografía por defecto, el número de columnas (\emph{onecolumn}, \emph{twocolumn}), y un largo etcétera de opciones avanzadas.

\section{Estructura de un documento}

Los documentos se estructuran en capítulos (\textbf{\emph{\textbackslash chapter\{nombre\}}}), secciones (\textbf{\emph{\textbackslash section\{nombre\}}}), subsecciones (\textbf{\emph{\textbackslash subsection\{nombre\}}}), subsubsecciones (\textbf{\emph{\textbackslash subsubsection\{nombre\}}}), subsubsub...section (\textbf{\emph{\textbackslash  subsubsub...section \{nombre\}}}). Si tras cualquiera de estos elementos ponemos un * (e.g.: \emph{chapter*\{Nombre capítulo\}}), entonces dicho elemento no se numerará y no se incluirá por defecto en ningún índice.

Según el tipo de documento, puede que algunos de estos elementos estructurales no estén disponibles, y en cambio estén disponibles otros en su lugar.

Por ejemplo, el tipo de documento \emph{beamer} permite crear con \LaTeX presentaciones de diapositivas. En este caso no se aceptaría la división por capítulos, y en su lugar se permitiría la definición de diapositivas entre los comandos \textbf{\textbackslash begin\{frame\} ... \textbackslash end\{frame\}}.

En el tipo de documento \emph{book} se permite también la creación de anexos como capítulos especiales definidos entre los comandos \textbf{\textbackslash begin\{appendices\} ... \textbackslash end\{appendices\}}.

Por otro lado, en el tipo \emph{book} podemos especificar qué partes de nuestro documento están orientados al título, abstract, índices, etc., al contenido principal y a la bibliografía, glosarios, etc. Para ello, usaremos, respectivamente el comando \textbackslash frontmatter, \textbackslash mainmatter y \textbackslash backmatter justo antes del comienzo de cada uno de estos apartados de nuestro documento. Estos comandos permiten que los números de páginas y el estilo general de las páginas sean diferentes según la parte del documento que estemos especificando.

Finalmente, podremos siempre usar el comando \textbackslash clearpage para forzar un salto de página.

\section{Paquetes}

Antes hemos mencionado los ``paquetes''. Un paquete es un archivo de \LaTeX que incluye nuevos estilos, comandos, opciones, etc. para dar un formato apropiado a nuestros documentos. Hay multitud de dichos paquetes, y es imprescindible conocer algunos de ellos para poder escribir un documento.

Un paquete se incluye en el preámbulo mediante la palabra \textbf{\textbackslash usepackage}. Podemos observar en el código de este documento los paquetes más básicos que se usan en casi cualquier documento \LaTeX. 

El nombre del paquete será el indicado entre \{ \}.  Entre [ ] se introducen las opciones para configurar dicho paquete.

Los, literalmente, cientos de paquetes disponibles en \LaTeX se pueden consultar en: \url{https://ctan.org/pkg}.

% obsérvese como podemos usar el comando \url{} para incluir urls en nuestros textos. LaTeX generará los hipervínculos automáticamente y le pondrá un estilo diferente.

Es bueno que al incluir un paquete dejemos un pequeño comentario en nuestro código para indicar para qué sirve dicho paquete.

\section{Estilos básicos}

En \LaTeX podemos,  al igual que en Word, escribir en \textbf{negrita}, \emph{cursiva} o \underline{subrayar}.

Para crear párrafos formateados adecuadamente debemos siempre recordar dar dos intros en nuestro código.

Con \LaTeX es bueno atender a los detalles de estilo para conseguir unos resultados muy \textbf{\underline{\emph{profesionales}}}. La forma alternativa de introducir las dobles comillas ``'' es un ejemplo de ello.

\section{Tamaños y tipos de fuentes}

En el preámbulo podemos especificar el tamaño por defecto de la fuente para todo el documento (comando \emph{documentclass}). No obstante, podemos especificar cualquiera de los siguientes tamaños por defecto para nuestros textos (una vez que ponemos un tamaño de letra, continuará dicho tamaño hasta que no pongamos otro):

\Huge Texto, \huge  Texto, \LARGE Texto, \Large Texto, \large Texto, \normalsize Texto (por defecto), \small Texto, \footnotesize Texto, \scriptsize Texto, \tiny Texto

\normalsize % es necesario poner esto para que se vuelva a poner el tamaño por defecto del documento

Asimismo, podemos especificar tamaños personalizados de fuente y de interlineado:

% en LaTeX se pueden especificar los tamaños mediante puntos, cm, mm, etc.
\fontsize{22pt}{1cm}\selectfont Texto
% el primer parámetro es el tamaño de la tipografía y el segundo el tamaño de interlineado.

\normalsize % es necesario poner esto para que se vuelva a poner el tamaño por defecto del documento

Igualmente podemos especificar el tipo de fuente: 

\textrm{Esta frase tiene una tipografía \emph{serif} (por defecto)}

\textsf{Esta frase tiene una tipografía \emph{sans serif}}

\texttt{Esta frase tiene una tipografía \emph{typewriter}}

O bien especificarla para todo nuestro documento en el preámbulo mediante el comando: % mirar el comando en el PDF

\textbf{\textbackslash renewcommand\{\textbackslash familydefault\}\{\textbackslash sfdefault\}}

Los tipos de fuentes disponibles para el preámbulo son: \textbackslash rmdefault (serif, por defecto), \textbackslash sfdefault (sans serif), \textbackslash ttdefault (typewriter)

Mediante paquetes adicionales se dispone de una gran variedad de tipografías (tantas como en cualquier procesador de textos).

\section{Colores}

\subsection{Texto}

Al incluir el paquete ``xcolor'' en el preámbulo podremos colorear  nuestro texto. Para ello usaremos el comando \textbf{\textbackslash textcolor\{\emph{nombre color}\}}, siendo el nombre del color cualquiera de los disponibles de manera predeterminada o cualquier definido por nosotros en el preámbulo del documento.

Veamos un ejemplo:

\textcolor{red}{Praesent in sapien}.
\textcolor{blue}{Lorem ipsum dolor sit amet, consectetuer 
adipiscing elit}.
\textcolor{TealBlue}{Duis fringilla tristique neque}.
\textcolor{GreenYellow}{Sed interdum libero ut metus}.
\textcolor{gray}{Pellentesque placerat. Nam rutrum augue a leo}. 
\textcolor{pink}{Morbi sed elit sit amet ante lobortis sollicitudin.}

Podemos consultar el listado completo de colores básicos y avanzados en: \url{https://es.overleaf.com/learn/latex/Using_colours_in_LaTeX#Reference_guide}.

Si algún color no está disponible en dicho listado, o deseamos especificar uno muy concreto, podemos usar el comando \textbf{\textbackslash definecolor\{\emph{nombre nuevo color}\}\{\emph{formato (e.g.: RGB)}\}\{\emph{valores (e.g.: 219, 48, 122}\}} en el preámbulo del documento. A partir de ese momento podremos usar ese nombre de color para especificar dicho color en todo nuestro documento.

\subsection{Fondo de páginas y color por defecto}

El color de fondo se especifica mediante el comando \textbf{\textbackslash pagecolor\{nombre color\}} y el color del texto por defecto mediante  \textbf{\textbackslash color\{nombre color\}}.

\section{Alineación de textos}

\begin{flushleft}
El texto lo podemos escribir alineado a la izquierda,	
\end{flushleft}

\begin{center}
centrado
\end{center}

\begin{flushright}
o alineado a la derecha,	
\end{flushright}

\section{Notas al pie y al margen}

\LaTeX permite crear notas al pie o al margen de página de manera muy simple, usando los comandos \textbf{\textbackslash foonote\{nota al pie\}} y \textbf{\textbackslash marginpar\{nota al margen\}}.

Por ejemplo, este párrafo tiene asociado una nota al pie de página\footnote{Las notas al pie de página en \LaTeX son muy sencillas de incluir}.

Este otro tiene asociado una nota al margen (podemos observar como las notas al margen no están numeradas)\marginpar{Las notas al margen de página en \LaTeX son muy sencillas de incluir}.

\section{Listas}

Existen dos tipos de listas: Numeradas y no numeradas. Un ejemplo de ambas es el siguiente:

\begin{enumerate}
\item Primero
\item Segundo
\end{enumerate}

\begin{itemize}
\item Primero
\item Segundo
\end{itemize}

Además, se pueden anidar:

\begin{itemize}
\item Primero
	\begin{enumerate}
	\item A
	\item B
	\end{enumerate}
\item Segundo
	\begin{enumerate}
	\item A
	\item B
	\end{enumerate}
\end{itemize}

\subsection{Estilos en listas no numeradas}

Podemos personalizar los estilos de las listas mediante mediante el siguiente código en el preámbulo (se cambiaría el estilo de todas las listas de todo el documento) o justo antes de comenzar una lista (solo se aplica a partir de dicha lista en adelante):

% Fijarse en el resultado del PDF para la siguiente línea
\textbf{\textbackslash renewcommand \textbackslash labelitemi\{\emph{ESTILO}\}}

En la definición de estos estilos podemos especificar, en vez de ``labelitemi'':

\begin{itemize}
	\item \emph{labelitemii} para listas no numeradas de Nivel 2
	\item \emph{labelitemiii} para listas no numeradas de Nivel 3
	\item\emph{labelitemiv} para listas no numeradas de Nivel 4
\end{itemize}

Entre los estilos que podemos especificar están:

% Fijarse en el resultado del PDF para las siguiente línea
\renewcommand\labelitemi{$\cdot$}
\begin{itemize}
	\item Cualquier texto o símbolo que deseemos, con el estilo que deseemos (negrita, cursiva, etc.).
	\item \$\textbackslash square\$: $\square$
	\item \$\textbackslash bullet\$: $\bullet$
	\item \$\textbackslash ast\$: $\ast$
	\item \$\textbackslash cdot\$: $\cdot$
	\item \$\textbackslash blacksquare\$: $\blacksquare$
\end{itemize}

% reiniciamos el estilo de las listas
\renewcommand\labelitemi{\tiny$\blacksquare$}

\subsection{Estilos en listas numeradas}

Podemos personalizar los estilos de las listas numeradas mediante mediante el siguiente código en el preámbulo (se cambiaría el estilo de todas las listas de todo el documento) o justo antes de comenzar una lista (solo se aplica para dicha lista):

% Fijarse en el resultado del PDF para la siguiente línea
\textbf{\textbackslash renewcommand \textbackslash labelenumi\{\emph{ESTILO}\}}

En la definición de estos estilos podemos especificar, en vez de ``labelenumi'':

\begin{itemize}
	\item \emph{labelenumii} para listas numeradas de Nivel 2
	\item \emph{labelenumiii} para listas numeradas de Nivel 3
	\item \emph{labelenumiv} para listas numeradas de Nivel 4
\end{itemize}

Entre los estilos que podemos especificar están:

% Fijarse en el resultado del PDF para las siguientes líneas
\renewcommand\labelenumi{\scriptsize\emph{$\Roman{enumi}$.}}
\begin{enumerate}

	\item \$\textbackslash arabic\{enumi\}\$\ - Estilo de números (estilo por defecto) (1, 2, 3, etc.).
	
	\item \$\textbackslash roman\{enumi\}\$\ - Estilo de números romanos (i, ii, iii, etc.). Poner enumi, enumii, enumiii ó enumiv según el nivel de la lista.

	\item \$\textbackslash Roman\{enumi\}\$\ - Estilo de números romanos en mayúsculas (I, II, III, IV, etc.).

	\item \$\textbackslash alph\{enumi\}\$\ - Estilo de letras (a, b, c, etc.).

	\item \$\textbackslash Alph\{enumi\}\$\ - Estilo de letras mayúsculas (A, B, C, etc.).

\end{enumerate}

% reiniciamos el estilo de las listas
\renewcommand\labelenumi{$\arabic{enumi}$.}

También podemos especificar el número inicial del contador en nuestras listas numeradas. Para ello usaremos el comando \textbf{\textbackslash setcounter\{enumi\}\{\emph{VALOR INICIAL}\}} (pondremos enumi, enumii, etc. según deseemos) dentro de la definición de nuestra lista:

\begin{enumerate}
\setcounter{enumi}{4} % aquí aplicariamos el comando
	\item Primero
	\item Segundo
	\item Tercero
\end{enumerate}

\section{División en múltiples archivos}

Una de las grandes dificultades que nos encontramos cuando comenzamos a trabajar con \LaTeX es lo dificultoso que es navegar por el código de un documento extenso. Para evitar dicho problema, aunque es opcional, \LaTeX permite dividir un documento en diversos ficheros usando el comando \textbf{\textbackslash include\{\emph{RUTA ARCHIVO SIN EXTENSION .TEX}\}}. De esta forma, podemos tener ficheros para cada capítulo, sección, etc. La división en ficheros la podemos realizar como deseemos.

En este documento de ejemplo, los siguientes capítulos se encuentran en ficheros aparte, dentro de la carpeta \emph{capitulos}: \emph{cap02.tex} y \emph{cap03.tex}.

Como se puede observar, el documento final incluirá todo el texto y comandos tanto del fichero principal como de los ficheros adicionales.

\chapter{Tablas, figuras, ecuaciones e índices}

\section{Figuras}

Para poder incluir figuras en nuestros documentos es necesario incluir el paquete ``graphicx''. Además, es conveniente (aunque opcional) meter todas las figuras de nuestro documento en una carpeta dedicada a ello. \LaTeX permite predefinir una carpeta para incluir las figuras de todo nuestro documento usando el siguiente comando en el preámbulo: \textbf{\textbackslash graphicspath\{ \{\emph{CARPETA}\} \}}.

A diferencia de los procesadores de texto habituales, las figuras se \emph{referencian}, no se \emph{insertan} en el documento.

Para referenciar una figura se utiliza en comando \textbf{\textbackslash  includegraphics\{\emph{NOMBRE IMAGEN}\}} (el nombre de la imagen también puede hacer referencia a una subcarpeta). En el nombre de la imagen no será necesario incluir la extensión. Además, \LaTeX no solo permitirá imágenes PNG, JPEG, etc., sino también gráficos vectoriales en formato PDF, EPS, etc. De hecho, se recomienda usar \textbf{imágenes en formato vectorial siempre que sea posible}.

Veamos un ejemplo de imagen vectorial:

\includegraphics{LaTeX_vectorial}

Las figuras pueden también escalarse, rotarse, etc. según nuestras necesidades usando una variante del comando anterior: \textbf{\textbackslash includegraphics[OPCIONES] \{\emph{NOMBRE IMAGEN}\}}. 

Al igual que en el resto de comandos de \LaTeX, las opciones se separan por comas. Las opciones más habituales disponibles para este comando son:

\begin{itemize}
	\item \emph{width}: Ancho de la imagen (en cm, mm, pt, etc.). Existen dos anchos especiales: \textbackslash textwidth y \textbackslash columnwidth. Con estos anchos podemos hacer la imagen del mismo ancho que el texto o columna actual, o bien aplicar un factor sobre dicho ancho. Por ejemplo \textbackslash includegraphics [width=0.5\textbackslash textwidth]\{\emph{NOMBRE IMAGEN}\} incluirá una imagen con un ancho del 50\% del ancho total del texto.
	\item \emph{height}: Alto de la imagen.
	\item \emph{scale}: Factor de escala sobre el tamaño original de la imagen.
	\item \emph{angle}: Ángulo de rotación de la imagen en grados y en el sentido antihorario.
\end{itemize}

Veamos un ejemplo: % se recomienda "jugar" con estas opciones

\includegraphics[width=0.2\textwidth, angle=45]{LaTeX}

Se podrán observar como \LaTeX intentará que las imágenes se autoposicionen: se alterará el interlineado, se ampliarán márgenes, etc. Esto puede suponer un problema en diversos casos (sobretodo cuando tenemos restricciones en el número de páginas). En secciones posteriores aprenderemos como arreglar este problema.

\section{Tablas}

La definición de tablas en \LaTeX es quizás de las cuestiones más complejas que posee.

La manera más básica de definir una tabla es como sigue:

% mirar el código de prueba en el PDF, o en el siguiente párrafo del código
\textbf{\textbackslash begin}\{tabular\}\{ c \vline c \vline c \}

 Celda 1 \& Celda 2 \& Celda 3 \textbackslash\textbackslash
 
 \textbf{\textbackslash hline}
 
 Celda 4 \& Celda 5 \& Celda 6 \textbackslash\textbackslash
 
 Celda 7 \& Celda 8 \& Celda 9
  
\textbf{\textbackslash end}\{tabular\}

Cuya visualización sería la siguiente: % mirar en el PDF el resultado

% mirar atentamente este código para saber como crear una tabla en LaTeX
\begin{tabular}{ c|c|c }
 Celda 1 & Celda 2 & Celda 3 \\
 \hline
 Celda 4 & Celda 5 & Celda 6 \\  
 Celda 7 & Celda 8 & Celda 9    
\end{tabular}

Se puede observar como se deben de especificar las columnas primero, con la alineación de cada una:

\begin{itemize}
	\item \emph{c}: Columna centrada
	\item \emph{r}: Columna alineada a la derecha
	\item \emph{l}: Columna alineada a la izquierda
\end{itemize}

% nótese que las lineas verticales en LaTeX se escriben con \vline. El símbolo | se usa para crear líneas horizontales largas.
Las ``\vline'' permiten especificar los bordes en las columnas. Se pueden poner dobles o simples, externos o internos, y allá donde deseemos.

A nivel de contenidos, las diferentes columnas se separan con el símbolo \&. Las filas con el símbolo ``\textbackslash\textbackslash'', que en \LaTeX permite además crear un salto de línea.

Para dividir las filas, de manera opcional también podemos usar el comando \textbf{\textbackslash hline}, que crea una línea horizontal.

Podemos especificar líneas horizontales entre determinadas columnas únicamente usando el comando \textbf{\textbackslash cline\{COL. INICIAL - COL. FINAL\}}.

\subsection{Tamaños de columnas}

Para poder especificar tamaños de columnas fijos es necesario incluir en el preámbulo el paquete ``array''.

Podemos crear columnas con un tamaño fijo usando para ello, en la especificación de las columnas, el tipo especial \emph{m} de la siguiente forma:

\textbf{\textbackslash begin}\{tabular\}\{ m\{TAMAÑO\} \vline m\{TAMAÑO\} \vline m\{TAMAÑO\} \}

 Celda 1 \& Celda 2 \& Celda 3 \textbackslash\textbackslash
 
 \textbf{\textbackslash hline}
 
 Celda 4 \& Celda 5 \& Celda 6 \textbackslash\textbackslash
 
 Celda 7 \& Celda 8 \& Celda 9
  
\textbf{\textbackslash end}\{tabular\}

Cuya visualización sería la siguiente: % mirar en el PDF el resultado

% mirar atentamente este código para saber como crear una tabla en LaTeX con columnas de tamaño fijo
\begin{tabular}{ m{2.5cm}|m{25mm}|m{40pt}|m{0.15\textwidth} }
 Celda 1 & Celda 2 & Celda 3 & Celda 4 \\
 \hline
 Celda 5 & Celda 6 & Celda 7 & Celda 8 \\  
 Celda 9 & Celda 10 & Celda 11 & Celda 12   
\end{tabular}

\subsection{Combinación de filas y columnas}

Para poder combinar filas y columnas es necesario incluir el paquete ``multirow'' en el preámbulo.

Un ejemplo de tabla con filas y columnas combinadas es el que sigue\footnote{Mirar el código de este documento para entender cómo se especifican}:

\begin{tabular}{ |l||c|  }
 \hline
 \multicolumn{2}{|c|}{\textbf{Comunidades autónomas}} \\
 \hline
  \emph{Nombre} & \emph{Provincia}\\
 \hline
 % en las filas combinadas es necesario especificar el ancho de la columna o bien poner * para que se calcule automáticamente
 \multirow{8}{*}{Andalucía} & Granada\\
 \cline{2 - 2} % línea horizontal solo para la columna 2
 	& Almería\\
 \cline{2 - 2}
 	& Jaén\\
 \cline{2 - 2}
 	& Málaga\\
 \cline{2 - 2}
 	& Cádiz\\
 \cline{2 - 2}
 	& Córdoba\\
 \cline{2 - 2}
 	& Sevilla\\
 \cline{2 - 2}
 	& Huelva\\
 \hline
 \multirow{2}{*}{Extremadura} & Cáceres\\
 \cline{2 - 2} % línea horizontal solo para la columna 2
 	& Badajoz\\
 \hline
\end{tabular}

\subsection{Importación desde CSV}

\LaTeX permite crear tablas a partir de archivos CSV externos. Para ello tendremos que importar el paquete ``csvsimple''. Dicho paquete permite autogenerar las tablas mediante el comando \textbf{\textbackslash csvautotabular\{\emph{fichero CSV}\}} o bien usar los datos del CSV para generar nuestra propia tabla personalizada.

Por ejemplo, el comando \textbackslash csvautotabular\{csv/ejemplo.csv\} generaría la siguiente tabla:

\csvautotabular{csv/ejemplo.csv}

Si deseamos tener un control exhaustivo sobre cómo se visualizan los diferentes elementos de la tabla, podemos usar el comando \textbf{\textbackslash csvreader[opciones]\{\emph{fichero CSV}\}\{\emph{columnas}\}\{\emph{formato columnas}\}}. Las \emph{columnas} son una secuencia de ``nombre=\textbackslash clave'' donde a cada nombre de columna del CSV le asociaremos un comando de \LaTeX personalizado. En \emph{formato columnas} usaremos dichos comandos para aplicarle un formato a los valores de dichas columnas en cada fila. También podremos usar el comando \textbf{\textbackslash thecsvrow} para referirnos al número de la fila.

Un ejemplo de su uso se muestra a continuación\footnote{Véase el código \LaTeX para comprender cómo se ha generado la tabla mostrada}. La documentación completa del paqueta ``csvsimple'' se encuentra disponible en \url{https://tools.ietf.org/doc/texlive-doc/latex/csvsimple/csvsimple.pdf}.

\begin{tabular}{l|l||c|c|c}
	\hline
	\textbf{\#} & \textbf{Mes} & \textbf{1958} & \textbf{1959} & \textbf{1960}\\
	\hline
	\csvreader[head to column names, late after line=\\\hline]{csv/ejemplo.csv}{Month=\month,1958=\anhoA,1959=\anhoB,1960=\anhoC}{\textbf{\thecsvrow} & \month & \emph{\anhoA} & \emph{\anhoB} & \emph{\anhoC}}
 \end{tabular}
 
\section{Listados de código}

A lo largo de este documento hemos tenido que incluir código de \LaTeX para comprender su sintaxis. Existen dos comandos para hacer listados de código de manera mucho más sencilla a como hemos visto hasta ahora.

El primero de ellos es \textbf{\textbackslash begin\{verbatim\} ... \textbackslash end\{verbatim\}}, que permite incluir texto ``literal'' en nuestros documentos, sin que \LaTeX interprete ninguno de sus contenidos como un posible comando. El siguiente ejemplo ilustra el uso de este comando\footnote{Observar PDF y código}:

\begin{verbatim}
	\begin{tabular}{ m{2.5cm}|m{25mm}|m{40pt}|m{0.15\textwidth} }
 		Celda 1 & Celda 2 & Celda 3 & Celda 4 \\
 		\hline
 		Celda 5 & Celda 6 & Celda 7 & Celda 8 \\  
 		Celda 9 & Celda 10 & Celda 11 & Celda 12   
	\end{tabular}
\end{verbatim}

Obsérvese como \emph{verbatim} además usa un tipo de fuente especial para estos fragmentos de texto.

El otro comando, mucho más avanzado que \emph{verbatim}, es \textbf{\textbackslash begin\{lstlisting\} ... \textbackslash end\{lstlisting\}}. Dicho comando requiere incluir el paquete ``listings'' en el preámbulo del documento. Este comando se diferencia de \emph{verbatim} en que permite incluir ficheros externos, realzar partes del código con colores o tipos de letras diferentes, etc. Además, incluye soporte por defecto a multitud de lenguajes de programación, aunque podemos especificar nuestro propio esquema de realzado para cualquier texto o lenguaje. El listado de lenguajes soportados por defecto podemos consultarlo en: \url{https://es.overleaf.com/learn/latex/Code_listing#Reference_guide}.

El siguiente ejemplo ilustra el uso de este comando para mostrar un código del lenguaje R\footnote{Observar PDF y código}:

\begin{lstlisting}[language=R]
if(num < 0) {
   print("El factorial no existen para numeros negativos")
} else if(num == 0) {
   print("El factorial de 0 es 1")
} else {
   for(i in 1:num) {
      factorial = factorial * i
   }
   print(paste("El factorial de", num ,"es",factorial))
}
\end{lstlisting}

El ejemplo que se muestra a continuación incluye un código R desde un fichero externo (carpeta ``codigo\_externo/ejemplo.R''):

\lstinputlisting[language=R]{codigo_externo/ejemplo.R}

Entre las opciones de este comando de importación de código externo están \emph{firstline} y \emph{lastline}, que permiten especificar el rango de líneas de código a mostrar en el documento.
	
\section{Ecuaciones}

El soporte a ecuaciones en \LaTeX es ampliamente conocido. De hecho, su funcionalidad es tan amplia y avanzada que hay personas que piensan que \LaTeX es un lenguaje exclusivamente para ``matemáticos'' (¡nada más lejos de la realidad!).

Para crear una ecuación en línea simplemente deberemos escribir dicha ecuación entre \textbf{\$}. Por ejemplo el código \$x\^{}2+x+1=y\_1\$ generaría como resultado\footnote{véase código y PDF a la vez} $x^2+x+1=y_1$.

Si las ecuaciones se escriben entre \textbf{\$\$}, entonces \LaTeX reservará una línea específica para dichas ecuaciones y unos márgenes superiores e inferiores.

Veamos algunos ejemplos para tener una referencia de la escritura básica de ecuaciones (dicha escritura esta fuera del ámbito de esta guía):

$$[ x^n + y^n = z^n ]$$
$$( x_1+x_2=4 )$$
$$E=mc^2$$
$$2\times 2 = 4$$
$$2\div 2 = 4$$
$$\frac{1}{2} = 0.5$$
$$\sqrt{\pi}$$
$$x \in \Re$$
$$\prod_{i=a}^{b} f(i) $$
$$\sum_{n=1}^{\infty} 2^{-n} = 1 $$
$$\lim_{x\to\infty} f(x)$$
$$\binom{n}{k} = \frac{n!}{k!(n-k)!}$$
% El comando \, permite poner un espacio "a mano" en nuestras ecuaciones
$$\int_0^1 f(x) \,dx$$
$$\oint_V f(s) \,ds$$
$$\begin{pmatrix}
1 & 2 & 3\\
a & b & c
\end{pmatrix}$$

Es importante notar que \LaTeX tiene soporte para todas las letras griegas dentro de las ecuaciones, usando para ello ``\textbackslash nombre letra''. Por ejemplo, ``\textbackslash beta'' generaría la letra griega $\beta$. Veamos algunos ejemplos\footnote{Mirar detenidamente el código y el resultado en el PDF}:

$$\alpha^n_1 + \beta^n = \rho_{n,1}$$
$$\int_\epsilon^\gamma x^n \,dx = \frac{\gamma^{n+1}}{n+1} - \frac{\epsilon^{n+1}}{n+1} + C$$

\LaTeX además incluye infinidad de comandos adicionales para operadores, símbolos, matrices, etc. Muchos de ellos se pueden consultar en \url{https://es.overleaf.com/learn/latex/List_of_Greek_letters_and_math_symbols}.

\section{Posicionamiento}

Como hemos observado previamente, \LaTeX es capaz de posicionar automáticamente nuestras figuras o tablas. Sin embargo, en muchas ocasiones es necesario realizar dichos posicionamientos manualmente. En las siguientes subsecciones aprenderemos cómo hacer esto.

Para posicionar nuestras figuras deberemos incluirlas dentro entre los comandos \textbf{\textbackslash begin\{figure\}[\emph{OPCIONES POSICION}] ... \textbackslash end\{figure\}}.

Por ejemplo, podríamos usar el siguiente código para posicionar manualmente una de las imágenes que hemos usado previamente en nuestros ejemplos:

% language={[LaTeX]Tex} realza la sintáxis del código LaTeX automáticamente
\begin{lstlisting}[language={[LaTeX]Tex}]
\begin{figure}[h]
  \includegraphics{LaTeX_vectorial}	
\end{figure}
\end{lstlisting}

Cuyo resultado sería:

\begin{figure}[h]
	\includegraphics{LaTeX_vectorial}	
\end{figure}

Entre las opciones de posicionamiento se encuentran las siguientes:

\begin{itemize}
	\item \emph{h}: Posicionar ``aquí'', posiciona, si se puede, en el mismo lugar del texto donde ponemos nuestro código de inserción de la figura.
	\item \emph{t}: Posicionar en la parte superior de la página actual.
	\item \emph{b}: Posicionar en la parte inferior de la página actual.
	\item \emph{p}: Posicionar en una página especial al final de todo el documento, junto al resto de figuras que posicionemos ahí.
\end{itemize}

Podemos combinar las opciones previas. Por ejemplo podemos poner como opción \emph{hb}, para indicar que se posicione la figura exáctamente donde hemos puesto el código de inserción, pero si no se pudiese, que entonces se posicione en la parte inferior de la página.

También podemos agregar el símbolo ``!'' al final de las opciones de posicionamiento, para indicar que \LaTeX evite hacer cualquier posicionamiento automático en ningún caso. Por ejemplo, la opción ``h!'' pondría en cualquier caso la figura justo en el lugar donde hemos escrito el código de inserción de la misma.

Para posicionar nuestras tablas, será necesario crearlas entre los comandos \textbf{\textbackslash begin\{table\}[OPCIONES POSICION] ... \textbackslash end\{table\}}. Por lo demás, es equivalente al posicionamiento de figuras.

Adicionalmente, podemos usar \textbf{\textbackslash begin\{wrapfigure\} ... \textbackslash end\{wrapfigure\}} ó \textbf{\textbackslash begin\{wraptable\} ... \textbackslash end\{wraptable\}}, para crear figuras o tablas, respectivamente, que se posicionen de tal manera que el texto del documento las envuelva. Estos comandos estarán disponibles una vez incluyamos el paquete ``wrapfig'' en el preámbulo del documento.

Por ejemplo, si usamos el siguiente código:

\begin{lstlisting}[language={[LaTeX]TeX}]
Praesent in sapien. Lorem ipsum dolor sit amet, consectetuer 
adipiscing elit. Duis fringilla tristique neque. Sed interdum 
libero ut metus. Pellentesque placerat. Nam rutrum augue a leo. 
Morbi sed elit sit amet ante lobortis sollicitudin.

\begin{wraptable}{r}{8cm}
  \begin{tabular}{ c|c|c|c }
    Celda 1 & Celda 2 & Celda 3 & Celda 4 \\
    \hline
    Celda 5 & Celda 6 & Celda 7 & Celda 8
  \end{tabular}
\end{wraptable}

Praesent in sapien. Lorem ipsum dolor sit amet, consectetuer 
adipiscing elit. Duis fringilla tristique neque. Sed interdum 
libero ut metus. Pellentesque placerat. Nam rutrum augue a leo. 
Morbi sed elit sit amet ante lobortis sollicitudin.
\end{lstlisting}


Obtendremos como resultado:

\emph{Praesent in sapien. Lorem ipsum dolor sit amet, consectetuer 
adipiscing elit. Duis fringilla tristique neque. Sed interdum 
libero ut metus. Pellentesque placerat. Nam rutrum augue a leo. 
Morbi sed elit sit amet ante lobortis sollicitudin.}

\begin{wraptable}{r}{8cm}
\begin{tabular}{ c|c|c|c }
 		Celda 1 & Celda 2 & Celda 3 & Celda 4 \\
 		\hline
 		Celda 5 & Celda 6 & Celda 7 & Celda 8
\end{tabular}
\end{wraptable}

\emph{Praesent in sapien. Lorem ipsum dolor sit amet, consectetuer 
adipiscing elit. Duis fringilla tristique neque. Sed interdum 
libero ut metus. Pellentesque placerat. Nam rutrum augue a leo. 
Morbi sed elit sit amet ante lobortis sollicitudin.}

Nótese como a continuación de \textbackslash begin\{wrapXXX\} hay que especificar el posicionamiento y el tamaño de la figura o tabla, siendo las opciones de posicionamiento: 
\begin{itemize}
	\item \emph{r}: derecha
	\item \emph{l}: izquierda
	\item \emph{c}: centro
	\item \emph{i}: interno
	\item \emph{o}: externo
\end{itemize}

\section{Leyendas}

\subsection{Figuras o tablas}

Para poner una leyenda a nuestra figura o tabla, simplemente se utilizará el comando \textbf{\textbackslash caption\{\emph{LEYENDA}\}} entre los comandos \textbackslash begin\{figure\} ... \textbackslash end\{figure\}, \textbackslash begin\{table\} ... \textbackslash end\{table\}, \textbackslash begin\{wrapfigure\} ... \textbackslash end\{wrapfigure\} ó \textbackslash begin\{wraptable\} ... \textbackslash end\{wraptable\}.

\LaTeX se encargará de numerar adecuadamente las leyendas, aplicarles un estilo, posicionarlas, etc.

Por ejemplo, el código:

\begin{lstlisting}[language={[LaTeX]Tex}]
\begin{figure}[h]
  \centering
  \includegraphics[width=0.3\textwidth]{LaTeX_vectorial}
  \caption{Leyenda de una figura}	
\end{figure}

\begin{figure}[h]
  \caption{Leyenda de otra figura (arriba)}	
  \centering
  \includegraphics[width=0.3\textwidth]{LaTeX_vectorial}
\end{figure}
\end{lstlisting}

Generará como resultado:

\begin{figure}[h]
	\centering
	\includegraphics[width=0.3\textwidth]{LaTeX_vectorial}
	\caption{Leyenda de una figura (debajo)}	
\end{figure}

\begin{figure}[h]
  \caption{Leyenda de otra figura (encima)}	
  \centering
  \includegraphics[width=0.3\textwidth]{LaTeX_vectorial}
\end{figure}

Nótese como se ha utilizado el comando \textbf{\textbackslash centering}, muy habitual al especificar leyendas, para centrar la figura.

\subsection{Subfiguras}

Entre los comandos \textbf{\textbackslash begin\{figure\}[OPCIONES POSICION] ... \textbackslash end\{figure\}} podemos especificar más de una figura a modo de ``subfiguras'', cada una con su propia leyenda. Para ello, primeramente deberemos incluir el paquete ``subcaption'' dentro del preámbulo.

Por ejemplo, el código:

\begin{lstlisting}[language={[LaTeX]Tex}]
\begin{figure}[h]
  \begin{subfigure}{0.5\textwidth}
    \includegraphics[width=0.9\linewidth]{LaTeX_vectorial}
    \caption{Leyenda 1}
  \end{subfigure}
  \begin{subfigure}{0.5\textwidth}
    \includegraphics[width=0.9\linewidth]{LaTeX_vectorial}
    \caption{Leyenda 2}
  \end{subfigure}
	
  \begin{subfigure}{\textwidth}
    \centering
    \includegraphics[width=0.8\linewidth]{LaTeX_vectorial}
    \caption{Leyenda 3}
  \end{subfigure}

  \caption{Leyenda para las tres figuras}
\end{figure}
\end{lstlisting}

Generaría el siguiente resultado:

\begin{figure}[h]
	\begin{subfigure}{0.5\textwidth}
		\includegraphics[width=0.9\linewidth]{LaTeX_vectorial} 
		\caption{Leyenda 1}
	\end{subfigure}
	\begin{subfigure}{0.5\textwidth}
		\includegraphics[width=0.9\linewidth]{LaTeX_vectorial}
		\caption{Leyenda 2}
	\end{subfigure}
	% si se pone un intro adicional entre dos subfigures, entonces se pasa a la siguiente línea
	
	\begin{subfigure}{\textwidth}
		\centering
		\includegraphics[width=0.8\linewidth]{LaTeX_vectorial}
		\caption{Leyenda 3}
	\end{subfigure}

	\caption{Leyenda para las tres figuras}
\end{figure}

\subsection{Código}

Para incluir una leyenda en un código especificado entre \textbf{\textbackslash begin\{lstlisting\} ... \textbackslash end\{lstlisting\}} ó \textbf{\textbackslash lstinputlisting} usaremos la opción ``caption'' de dichos comandos.

Por ejemplo el comando \textbackslash lstinputlisting[language=R, caption=Ejemplo] \{codigo\_externo/ejemplo.R\} generará:

\lstinputlisting[language=R, caption=Código R de ejemplo]{codigo_externo/ejemplo.R}

\section{Etiquetas y referencias cruzadas}

Después de escribir cualquier comando que genere una numeración (capítulos, secciones, leyendas, etc.) podremos escribir el comando especial \textbf{\textbackslash label\{\emph{nombre etiqueta}\}}. Dichas etiquetas nos permitirán referenciar los números generados por \LaTeX mediante el comando complementario \textbf{\textbackslash ref\{\emph{nombre etiqueta}\}}. De esta forma, si cambia el documento, se reestructura, etc., jamás tendremos que preocuparnos de revisar las numeraciones de las referencias cruzadas.

Por ejemplo, a continuación vamos a referenciar la Figura \ref{fig:refs} y la Sección \ref{sec:indices}. En el primer caso, hemos puesto la etiqueta ``fig:refs'' tras la leyenda de la figura. En el segundo caso hemos puesto la etiqueta ``sec:indices'' tras el comando para generar la sección ``Índices'' (\ref{sec:indices}).

\begin{figure}[h]
	\centering
	\includegraphics[width=0.2\textwidth]{LaTeX_vectorial}
	\caption{Figura de ejemplo}	
	\label{fig:refs}
\end{figure}

Para etiquetar ecuaciones será necesario usar los comandos \textbf{\textbackslash begin\{equation\} ... \textbackslash end\{equation\}} para escribir nuestras ecuaciones en vez de \$ ó \$\$. Dentro podremos incluir el comando \textbackslash label.

\section{Índices}
\label{sec:indices}

Existen principalmente tres comandos para generar índices:

\begin{itemize}
	\item \textbf{\textbackslash tableofcontents}: Índice de capítulos, secciones, etc.
	\item \textbf{\textbackslash listoffigures}: Índice de figuras.
	\item \textbf{\textbackslash listoftables}: Índice de tablas.
\end{itemize}

Como ejemplo, al final de este capítulo se han incluido un índice de figuras.

Estos comandos tienen multitud de opciones adicionales que pueden consultarse en \url{https://es.overleaf.com/learn/latex/Table_of_contents}, \url{https://es.overleaf.com/learn/latex/Indices} y \url{https://es.overleaf.com/learn/latex/Lists_of_tables_and_figures}.

Por otro lado, podemos agregar nuestros propias entradas de índices personalizadas con el comando \textbf{\textbackslash addcontentsline\{toc\}\{\emph{tipo entrada}\}\{\emph{título entrada}\}}. El tipo de entrada corresponde con \emph{chapter}, \emph{section}, \emph{subsection}, etc., y permitirá que \LaTeX formatee correctamente la entrada agregada.

\listoffigures



\chapter{Gestión bibliográfica}

\LaTeX permite crear una bibliografía, y gestionar las citas a las diferentes entradas bibliográficas. Existen dos maneras de crear dichas sección: manualmente o a partir de una base de datos bibliográfica externa.

\section{Entradas manuales y citas}

Para crear una sección de bibliografía manual deberemos incluir diversas entradas entre los comandos \textbf{\textbackslash begin\{thebibliography\}\{\emph{nº entradas}\} ... \textbackslash end\{thebibliography\}}.

Las entradas se crean mediante el comando \textbf{\textbackslash bibitem\{\emph{etiqueta}\} \emph{Entrada bibliográfica completa (autores, título, etc.)}}.

Por otro lado, para citar en nuestros documentos cualquier entrada se utilizará el comando \textbf{\textbackslash cite\{\emph{etiqueta/s}\}}. Las numeraciones de entradas, forma de citar, etc. lo gestiona \LaTeX automáticamente. Sin embargo, el estilo de las entradas bibliográficas hay que gestionarlo manualmente (orden de autores, formato de títulos, etc.).

\section{Bases de datos bibliográficas externas y estilos automáticos}

Podemos incluir una base de datos bibliográfica externa que siga el formato Bibtex (muy común en software de gestión bibliográfica y soportado por las principales editoriales). El comando para incluir una base de datos bibliográfica externa es \textbf{\textbackslash bibliography\{\emph{nombre archivo Bibtex}\}}.

La principal ventaja, además de que podemos tener nuestras propias bases de datos compartidas entre diferentes documentos, es que el estilo de las entradas bibliográficas lo gestionará automáticamente \LaTeX. Para definir el estilo de las entradas bibliográficas se utilizará el comando \textbf{\textbackslash bibliographystyle\{\emph{estilo}\}}.

Entre los estilos predeterminados de \LaTeX están\footnote{Hay infinidad más a través de paquetes especializados. También podemos definir nuestros propios estilos.}: \emph{abbrv}, \emph{acm}, \emph{abbrv}, \emph{alpha}, \emph{apalike}, \emph{IEEEtran}, \emph{plain}, \emph{siam}, \emph{unsrt}.

Al final de este documento se ha incluido una sección de bibliografía, extraída del fichero ``bibliografia.bib''\footnote{El formato de este tipo de bases de datos bibliográficas es ajeno a este documento. Puede usarse software como JabRef, Bibdesk, Mendeley, etc. para generar dichas bases de datos.}. Se recomienda observar la bibliografía generada. Si se desea consultar más información, puede acudirse a las referencias \cite{overleaf-url, latexcompanion}.

Se pueden observar diversas entradas bibliográficas de ejemplo a través de las siguientes citas: \cite{maths-latex}, \cite{lamport}.

Se podrá observar, si se abre el archivo Bibtex que \LaTeX incluirá únicamente las entradas bibliográficas que se citen en el texto, omitiendo las que no.

\bibliographystyle{IEEEtran}
\bibliography{bibliografia.bib}


\end{document}
